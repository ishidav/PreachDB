%$Id: m2e_mapping.tex,v 1.5 2009/04/15 03:11:54 depaulfm Exp $
\chapter{Mur$\phi$ to Erlang Mapping}

\section{Semantic Mapping}

The main syntactic construct of a $mur\phi$ model is $rule$. The $rule$ 
construct implements
the guarded-commands semantic of $mur\phi$. A $rule$ comprises a $guard$ and a set
of $actions$. A $guard$ is an expression that evaluates to true (we say then that the 
guard is enabled) or false. 
%A set of $actions$ is a set of statements that implements
An action changes a state.

a transition from a state to another. Statements are treated as ordinary constructs
that are 'easily' mapped to $functional$ $programming$ constructs. The important
semantic mapping, however, is the translation of $mur\phi$'s state and 
transition-relations encoding to $preach$.
Note that a state of a $mur\phi$ model is global. All state manipulation is atomic, 
however.
%The $rule$ construct enforces atomicity. When multiple rules' guard enable instead of fire, each
%rules' actions are atomically and non-deterministically executed --- 
%all this is saying is that each fired rule is executed in an arbitrary order.
A $preach$ model shall preserve this semantic when mapped from a $mur\phi$ model


Let's start by looking at how we encode a state in $preach$. Figure~\ref{fig:Sst} gives
a syntax tree for a $state$ description. Each state is represented by 
a \underline{record}\footnote{Every underline word refers to an Erlang data-type or
keyword} of identifiers.  Each identifier represents a system's state variable. A record may 
contain nested records.  The order of the \underline{record}'s identifiers is identical to 
the variable declaration order in the original model. A bound integer represents each 
'encoding'.

As in  $mur\phi$, we implement the transition relation in Erlang as a set of rules.
In Figure~\ref{TRst}, we define the syntax tree for the transition relation. 
Notice the similarity of $<rules>$, $<rule>$ and $<expr>$ with $mur\phi$.  However,
we restrict ourselves to binary relations; we do not implement rule aliasing and statements
are simple assignment statements.
In Figure~\ref{TRsimplerule}, we present an Erlang pseudo-code to compute the state transition.
We start by defining the state in terms of \underline{record}. State variables may be
initialized at the time of the declaration of the record, e.g., $id2$. We declare each rule
following the syntax defined in Fig.~\ref{TRst}. In particular, E($X\#recordName.any$) is 
a function that evaluates an expression over the $X\#recordName.any$ state variable.
The term $dc$ in the second branch of 'when' represents a $dontcare$. Finally, the transition
function accepts a state as its input and it returns a set of states (successors). Each rule 
in the $Rules$ set is applied to the state $state$.
%make sure BNF no comma there  define the tokens
\begin{figure}
\begin{verbatim} 
<state>    ::= \[ <record> \]
<record>   ::= <ID>[=<encoding>] { , <ID>[=<encoding>] } { , <record> }
<encoding> ::=  integer | \[ integer  {, integer } \] 
<ID>       ::=  letter { [_ | letter] }

\end{verbatim}
\caption{State Syntax Tree}
\label{fig:Sst}
\end{figure}


\begin{figure}
\begin{verbatim}
<rules>      ::= \[ {<rule> [, <rule>]} \]
<rule>       ::= <simplerule> | <startstate> | <ruleset> | <invariant> 
<simplerule> ::= fun(X) when <expr> -> <stmts> { ,<stmts> } ; 
                                       <stmts> { ,<stmts> } end.  
<expr>       ::=    ( expr )
	             | X\#<ID>.<ID>
	             | integer
	             | <expr> < <expr>
	             | <expr> <= <expr>
	             | <expr> > <expr>
	             | <expr> >= <expr>
	             | <expr> == <expr>
	             | <expr> != <expr>

<stmts>      ::= X\#<ID>\{{ <ID>=<encoding> 
                                { , <ID>=<encoding> } \}

<startstate> ::= \#<ID>\{ <ID>=<encoding> 
                                { , <ID>=<encoding> } \}
\end{verbatim}
\caption{Transition Relation Syntax Tree, Part 1}
\label{TRsta}
\end{figure}
\begin{figure}
\begin{verbatim}
<ruleset>    ::= ruleset(X) -> lists:map(fun(X) -> <body> end, X).
<body>       ::=   <f_atom> [ {< f_atom> ; } ] [<f_record> { <f_records> ; }] 
<f_atom>      ::= f(X) when is_atom(X, <ID>) ->
                                 <expr> -> 
                                    <stmts> { ,<stmts> } ;  
                                 true ->
                                   null
                                 end; 
<f_record>   ::= f(X) when is_record(X, <ID>) ->
                                  lists:map(fun(Y) -> 
                                                    <expr> -> 
                                                       <stmts> { ,<stmts> } ;  
                                                    true ->
                                                      null
                                                    end 
                                          end., 
                                          X#<ID>);
  

<invariant>  ::= \#<ID>\{ <ID>=<encoding> 
                                { , <ID>=<encoding> } \}
\end{verbatim}
\caption{Transition Relation Syntax Tree, Part b}
\label{TRstb}
\end{figure}

\begin{algorithm}
\begin{algorithmic}
%<transition> ::  lists:map (fun
\STATE -record($recordName$, \{$id1$, $id2 = value$, $...$ , $idM$\}).
\newline
\STATE Rule1 =  fun(X) when E(\(X\#recordName.any\)) $->$ 
\STATE \ \ \ \ \ \ \ \ \ \ \ \ \ \ \ \ \ \ \ \ \ \ \ \ \ \ \ \ \ 
X\#\(recordName\{some\_id = encoding\}\); 
\STATE \ \ \ \ \ \ \ \ \ \ \ \ \ \ \ \ \ \ \ \ \ \ \ \ \ \ \ \ \
(X) $->$ $dc$ end.
\STATE $\vdots$
\STATE RuleN =  fun(X) when E(\(X\#recordName.any\))  $->$ 
\STATE \ \ \ \ \ \ \ \ \ \ \ \ \ \ \ \ \ \ \ \ \ \ \ \ \ \ \ \ \ 
X\#\(recordName\{some\_id = encoding\}\); 
\STATE \ \ \ \ \ \ \ \ \ \ \ \ \ \ \ \ \ \ \ \ \ \ \ \ \ \ \ \ \
(X) $->$ $dc$ end.
\newline
\STATE Rules = [ Rule1, Rule2, ... , RuleN ].
\newline
\STATE $function$ transition (record: state) $->$ 
\STATE \ [ lists:map(fun(X) $->$ X(state) end., Rules) ].
\caption{Erlang-pseudocode: Transition Relation with $simplerule$}
\label{TRsimplerule}
\end{algorithmic}
\end{algorithm}


\begin{algorithm}
\begin{algorithmic}
%<transition> ::  lists:map (fun
\STATE \% Declare records representing state variables
\STATE -record($recordName1$, \{$id1$, $id2 = value$, $...$ , $idM$\}).
\STATE -record($recordName2$, \{$id1$, $id2 = value$, $...$ , $idM$\}).
\STATE $\vdots$
\STATE -record($recordNameN$, \{$id1$, $id2 = value$, $...$ , $idM$\}).
\newline
\STATE \% Define record variables
\STATE Rec1  =  $\#recordName1\{\}.$ 
\STATE Rec2  =  $\#recordName1\{\}.$ 
\STATE $\vdots$
\STATE RecN  =  $\#recordNameN\{\}.$ 
\newline
\STATE \% Declare rule operations for each record 
\STATE each(X) when is\_record(X,$recordName1$) $->$
\STATE \ \ \ \ \ \ \ \ \ \ \ \ \ if is\_list(X\#$recordName1.id1$) $->$
\STATE \ \ \ \ \ \ \ \ \ \ \ \ \ \ \ \ \ lists:map( fun(Y) $->$ 
                           do\_something(Y) end, X\#$recordName1.id1$);          
\STATE \ \ \ \ \ \ \ \ \ \ \ \ \ true $->$ do\_something(X\#$recordName1.id1$)
\STATE \ \ \ \ \ \ \ \ \ \ \ \ \ end;
\STATE each(X) when is\_record(X,$recordName2$) $->$ ... ;
\STATE $\vdots$
\STATE each(X) when is\_record(X,$recordNameN$) $->$ ... .
\newline
\STATE \% Define state as a list of records
\STATE State =  [Rec1, Rec2, ..., RecN].
\newline
\STATE \% Define rule set, returning the set of next states
\STATE Ruleset1(State) -> lists:map(fun(X) $->$ each(X), State).
\caption{Erlang-pseudocode: Transition Relation with $ruleset$}
\label{TRrulseset}
\end{algorithmic}
\end{algorithm}



% OUT OF PLACE
%The $mur\phi$ model description is manually translated to a flat $mur\phi$ description
%so that procedures and functions are replaced by the actual statements they implement.

%\subsection{Data Types}
%A tuple represents a state

%\subsection{Procedures}
%\subsection{Functions}
%\subsection{Rules}

%\subsection{Statements}
%All statements are supported except $Undefine/clear$ since 

%-----------------------------------------------------------------
% Revision History 
% $Log: m2e_mapping.tex,v $
% Revision 1.5  2009/04/15 03:11:54  depaulfm
% Added new definitions for state and transition system; worked mainly on the figures; text still needs work
%
% Revision 1.4  2009/04/11 00:44:10  depaulfm
% Checking in changes last made on 2009-Mar-02
%
% Revision 1.3  2009/02/23 18:50:58  depaulfm
% 1st cut at pseudocode for TR
%
% Revision 1.2  2009/02/18 05:14:58  depaulfm
% Updated text to include a top-down description of the translation between murphi and preach
%
% Revision 1.1  2009/02/14 04:01:25  depaulfm
% Bootstrapping documentation for preach
%



%              [  f(X) when is_record(X, <ID>) ->
%                               lists:map(fun(Y) -> 
%                                                    <expr> -> 
%                                                       <stmts> { ,<stmts> } ;  
%                                                     true ->
%                                                       null
%                                                     end 
%                                           end., 
%                                           X#<ID>);
%                     |{f(X) when is_record(X, <ID>) ->
%                                lists:map(fun(Y) -> 
%                                                     <expr> -> 
%                                                        <stmts> { ,<stmts> } ;  
%                                                     true ->
%                                                       null
%                                                     end 
%                                           end., 
%                                           X#<ID>);
%                      }
%                    ]
%                    [  f(X) when is_atom(X, <ID>) ->
%                                <expr> -> 
%                                   <stmts> { ,<stmts> } ;  
%                                true ->
%                                  null
%                                end; 
%                     |{f(X) when is_atom(X, <ID>) ->
%                                <expr> -> 
%                                   <stmts> { ,<stmts> } ;  
%                                true ->
%                                  null
%                                end; 
%                      }
  
%
%
